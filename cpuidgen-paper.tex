% Encoding: UTF-8
% Build with: pdflatex
% Bibliography engine: bibtex8
% Should be a minimal set of packages and syntax bloat to be able to convert to HTML, RTF etc

\documentclass[a4paper,10pt,oneside,unicode]{article}
\usepackage[top=2cm,left=2.5cm,right=2cm,bottom=3cm]{geometry}
\usepackage[T2A]{fontenc}
\usepackage[utf8]{inputenc}
\usepackage{amsfonts}
\usepackage{amsmath}
\usepackage{amssymb}
\usepackage[english]{babel} % no Russian
\usepackage[pdftex]{graphicx}
\usepackage[pdftex]{hyperref}
\usepackage{indentfirst}

\hypersetup{colorlinks=true, linkcolor=black, filecolor=black, urlcolor=black,pdfauthor=Grigory Rechistov <grigory.rechistov@phystech.edu>,pdftitle=CPUID Simulation of Intel Processors}

\author{Grigory Rechistov\thanks{\texttt{grigory.rechistov@phystech.edu}} \and Name Surname\thanks{\texttt{email@mail.com}}}
\title{CPUID Simulation of Intel Processors}
\date{\today}

\newcommand{\abbr}{\emph{англ. }{}}
\newcommand{\todo}[1][]{\textcolor{red}{TODO #1}}
\newcommand{\othercopyright}{\textsuperscript{*}}
\begin{document}

\maketitle

\tableofcontents

\section {Introduction}

\todo{Write me}

\subsection{Contributions}
In this paper we make the following contributions.
\begin{enumerate}
\item Evaluate and compare existing means of processor features identification of different architectures.
\item Describe, implement and evaluate a structured solution to the simulation of CPUID instruction of Intel IA-32.
\end{enumerate}

\section{Overview of Processor Identification}

\subsection{MIPS}
\subsection{ARM}
\subsection{PowerPC}

\subsection{Intel IA-64 (Itanium)}

\cite{itanium-sdm}

\subsection{Intel IA-32 and Intel 64}

The common PC architecture, starting from Intel Pentium and its clones, provides \texttt{CPUID}~\cite{intelmanual-7vols} instruction. 

There is a number of complications that have resulted from long uncontrolled expansion of the CPUID

\paragraph{Elements adressing}

\begin{itemize}
\item Leaves
\item Subleaves
\item Registers
\item Bit range
\end{itemize}


\paragraph{Non-constant values} Firmware is able to suppress certain features indicated by CPUID by manipulating bits of model specific register (MSR) IA32\_MISC\_ENABLE. For example: \todo{NX, Leaf3, 1GB pages}

\paragraph{Topology-varaible elements}

Finally, it should be noted that, besides EAX, EBX, ECX, EDX, one more register may be affected by CPUID, namely IA32\_SIGNATURE \todo.

\section{Existing Approaches to CPUID Simulation}

What is required from a CPUId model.
\begin{itemize}
\item Be accurate \todo
\item Be configurable \todo
\end{itemize}

\subsection{Bochs}

\subsection{Xen}

\subsection{Qemu}

\subsection{Simics}

\section{The Structured Approach}

\section{Evaluation}

\section{Conclusions}

\section{Acknowledgements}

\bibliographystyle{ieeepes}
\bibliography{art} % use bibtex8, NO SPACES HERE!

\end{document}
